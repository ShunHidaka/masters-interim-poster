
標準シフト線型方程式 \eqref{eq-std-sls} は,量子力学や電子構造計算などに現れる.  
特に近年では,行列関数の計算における部分問題としても重要性が増している.

こうした問題では,\textcolor{red}{$10^7$~$10^8$次元を超える超大規模行列}が登場することもあり,従来の逐次的な解法では対応が困難である.

このような制約下では,次のような特性を持つアルゴリズムが求められる:
\begin{itemize}\setlength{\itemsep}{0pt}
  \item Matrix--free(行列全体を保持しない)
  \item 複数シフトに対する同時解法
  \item 並列計算への適用可能性
\end{itemize}
このような要請に応える手法として,Krylov部分空間法が注目されている.
