

CED上で2種類の並列化モデルおよび逐次実装の実行時間(秒)を比較した.
\begin{itemize} \setlength{\itemsep}{0pt}
	\item 問題設定
		\begin{itemize} \setlength{\itemsep}{0pt}
			\item CPU: Intel(R) Core(TM) i7-12700 12コア20スレッド
			\item VCNT4000std(4000次実対称),VCNT40000std(40000次実対称)\cite{ref-ELSES-matrix}
			\item $\sigma^{(k)} = 0.01 \exp(2\pi(k-0.5)/M),\ M=50$(実験1)
			\item $\sigma^{(k)} = -0.1+k*0.2/M + 0.01\mathrm{i},\ M=100$(実験2)
			\item 収束条件: 相対残差 $\|\vb{r}_j\| / \|\vb{r}_0\| \leq 10^{-13}$
			\item MPI: 4並列,OpenMP: --並列
		\end{itemize}
\end{itemize}
%%%%%
\vspace{0.2\baselineskip}
\begin{itemize}
	\item 実験1
\end{itemize}
\vspace{-0.7\baselineskip}
\begin{table}
	\begin{minipage}[b]{0.48\textwidth}
	\centering
		\caption*{VCNT4000std}
		\vspace{-8pt}
		\begin{tabular}{>{\centering\arraybackslash}p{5.8cm}>{\centering\arraybackslash}p{5.8cm}>{\centering\arraybackslash}p{5.8cm}}
			\hline
			逐次		& モデル1	& モデル2	\\ \hline
			3.44726	& 13.5996	& 0.988548	\\ \hline
		\end{tabular}
	\end{minipage}
	\hfill
	\begin{minipage}[b]{0.48\textwidth}
	\centering
		\caption*{VCNT40000std}
		\vspace{-8pt}
		\begin{tabular}{>{\centering\arraybackslash}p{5.8cm}>{\centering\arraybackslash}p{5.8cm}>{\centering\arraybackslash}p{5.8cm}}
			\hline
			逐次		& モデル1	& モデル2	\\ \hline
			676.268	& 156.539	& 912.709	\\ \hline
		\end{tabular}
	\end{minipage}
\end{table}
%%%%%
\vspace{0.5\baselineskip}
\begin{itemize}
	\item 実験2
\end{itemize}
\vspace{-0.7\baselineskip}
\begin{table}
	\begin{minipage}[b]{0.48\textwidth}
	\centering
		\caption*{VCNT4000std}
		\vspace{-8pt}
		\begin{tabular}{>{\centering\arraybackslash}p{5.8cm}>{\centering\arraybackslash}p{5.8cm}>{\centering\arraybackslash}p{5.8cm}}
			\hline
			逐次		& モデル1	& モデル2	\\ \hline
			2.55095	& 3.44825	& 0.267222	\\ \hline
		\end{tabular}
	\end{minipage}
	\hfill
	\begin{minipage}[b]{0.48\textwidth}
	\centering
		\caption*{VCNT40000std}
		\vspace{-8pt}
		\begin{tabular}{>{\centering\arraybackslash}p{5.8cm}>{\centering\arraybackslash}p{5.8cm}>{\centering\arraybackslash}p{5.8cm}}
			\hline
			逐次		& モデル1	& モデル2	\\ \hline
			160.371	& 32.1898	& 207.159	\\ \hline
		\end{tabular}
	\end{minipage}
\end{table}
%%%%%
\vspace{0.5\baselineskip}
\begin{itemize}
	\item 大規模行列では,通信の相対負荷が低下し,モデル1が最も高速だった
	\item 小規模行列では,通信のないモデル2が最も高速に動作した
	\item 超大規模行列では,モデル1がスケーラブルな解法として期待できる
\end{itemize}


